\begin{center}
\noindent\fbox{
  \parbox{0.95\linewidth}{
    \textbf{Definition:} \textbf{Satisfiable}. A compound proposition is satisfiable if there is an assignment of truth values to its variables
that makes it true (that is, when it is a tautology or a contingency).
  }
}
\end{center}
\begin{center}
\noindent\fbox{
  \parbox{0.95\linewidth}{
    \textbf{Definition:} \textbf{Unsatisfiable}. 
When no such assignments
exists, that is, when the compound proposition is false for all assignments of truth values to
its variables, the compound proposition is unsatisfiable.
  }
}
\end{center}

\begin{tcolorbox}[colback=white, colframe=gray!60, title=Remark 1]
Note that a compound proposition is
unsatisfiable if and only if its negation is true for all assignments of truth values to the variables,
that is, if and only if its negation is a tautology.
\end{tcolorbox}
\begin{tcolorbox}[colback=white, colframe=gray!60, title=Remark 2]
When we find a particular assignment of truth values that makes a compound proposition
true,we have shown that it is satisfiable; such an assignment is called a \textbf{solution} of this particular
satisfiability problem.
\end{tcolorbox}
\begin{tcolorbox}[colback=white, colframe=gray!60, title=Remark 3]
However, to show that a compound proposition is unsatisfiable, we need
to show that every assignment of truth values to its variables makes it false. Although we can
always use a truth table to determine whether a compound proposition is satisfiable, it is often
more efficient not to.
\end{tcolorbox}

\newpage
\begin{tcolorbox}[title=Example 1: Satisfiable compound propositions]
\textbf{Exercise:}  
\\ Determine whether each of the compound propositions is satisfiable
\begin{center}
$(p\lor \neg q) \land (q \lor \neg r) \land (r \lor \neg p)$ \\ 
$(p\lor q \lor r) \land (\neg p \lor \neg q \lor \neg r)$
\\
$(p\lor \neg q) \land (q \lor \neg r) \land (r \lor \neg p) \land (p \lor q \lor r) \land (\neg p \lor \neg q \lor \neg r)$
\end{center}

\textbf{Strategy:}  \\
Instead of using a truth table to solve this problem, we will reason about truth values. \\ \\
\textbf{First compound proposition:} \\
We have $(p\lor \neg q) \land (q \lor \neg r) \land (r \lor \neg p)$ is true when the three variables $p$, $q$, and $r$ have the same truth value.
Hence, it is satisfiable as there is at least
one assignment of truth values for $p$, $q$, and $r$ that makes it true.
\\ \\
\textbf{Second compound proposition:} \\
Similarly, note that $(p\lor q \lor r) \land (\neg p \lor \neg q \lor \neg r)$ is true when at least one of $p$, $q$, and $r$ is true and at least one is false. Hence, $(p\lor q \lor r) \land (\neg p \lor \neg q \lor \neg r)$ is satisfiable, as there is at least
one assignment of truth values for $p$, $q$, and $r$ that makes it true. \\ \\
\textbf{Third compound proposition:} \\
Finally, note that for $(p\lor \neg q) \land (q \lor \neg r) \land (r \lor \neg p) \land (p \lor q \lor r) \land (\neg p \lor \neg q \lor \neg r)$ to be true:
\\ \\ Both: 
\begin{center}
$(p\lor \neg q) \land (q \lor \neg r) \land (r \lor \neg p)$    
\end{center}
and 
\begin{center}
 $(p \lor q \lor r) \land (\neg p \lor \neg q \lor \neg r)$   
\end{center}
must both be true. 
\\ For the first to be true, $(p\lor \neg q) \land (q \lor \neg r) \land (r \lor \neg p)$ ,  the three variables must have the same truth values. \\
For the second to be true, $(p \lor q \lor r) \land (\neg p \lor \neg q \lor \neg r)$ at least one of the three variables must be true and at least one must be
false. \\ \\
However, these conditions are contradictory. From these observations we conclude
that no assignment of truth values to $p$, $q$, and $r$ makes $(p\lor \neg q) \land (q \lor \neg r) \land (r \lor \neg p) \land (p \lor q \lor r) \land (\neg p \lor \neg q \lor \neg r)$ true. Hence, it is unsatisfiable.
\end{tcolorbox}










¨.