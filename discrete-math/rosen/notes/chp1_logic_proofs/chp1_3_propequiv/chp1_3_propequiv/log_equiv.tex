\vspace{5pt}
\tcbset{colback=gray!5, colframe=black!70, boxrule=0.5pt, arc=2pt, left=6pt, right=6pt, top=4pt, bottom=4pt}
\begin{tcolorbox}[title=Definition: Logically Equivalent]
The compound propositions $p$ and $q$ are called \textbf{logically equivalent} if 
\begin{center}
$p\leftrightarrow q$ is a tautology
\end{center}
The notation $p\equiv q $ denotes that $p$ and $q$ are logically equivalent.
\end{tcolorbox}
\begin{tcolorbox}[colback=white, colframe=gray!60, title=Remark 1]
The symbol $\equiv$ is not a logical connective, and $p\equiv q$ is not a compound proposition
but rather is the statement that $p\leftrightarrow q$ is a tautology. The symbol $\Leftrightarrow$ is sometimes used instead
of $\equiv$ to denote logical equivalence.
\end{tcolorbox}
\tcbset{colback=gray!5, colframe=black!70, boxrule=0.5pt, arc=2pt, left=6pt, right=6pt, top=4pt, bottom=4pt}
\begin{tcolorbox}[title=Definition: De Morgan’s Laws]
A pair of transformation rules, that state:
\begin{center}
$\neg (p\land q) \equiv \neg p \lor \neg q$ \\
$\neg (p\lor q) \equiv \neg p \land \neg q$
\end{center}

They can also be expressed as:

\begin{center}
\rowcolors{2}{gray!10}{white}
\begin{tabular}{|
    >{\centering\arraybackslash}m{3cm}  % centered vertically + horizontally
    |>{\centering\arraybackslash}m{2.5cm}  % centered vertically + horizontally
    |m{3.5cm}  %  column, vertically centered
    |m{3.5cm}|}  %  column, vertically centered
\hline
\rowcolor{gray!20}
\textbf{Logic} & \textbf{Boolean} & \textbf{English (simple)} & \textbf{English (formal)} \\
\hline
$\neg (p\land q) \equiv \neg p \lor \neg q$ & 
$\overline{p \land q} = \overline{p} \lor \overline{q}$  & The negation of "A and B" is the same as "not A or not B" & The complement of the union of two sets is the same as the intersection of their complements \\
\hline
$\neg (p\lor q) \equiv \neg p \land \neg q$ & $\overline{p \lor q} = \overline{p} \land \overline{q}$ & The negation of "A or B" is the same as "not A and not B" & The complement of the intersection of two sets is the same as the union of their complements\\
\hline  
\end{tabular}
\end{center}
\end{tcolorbox}

\begin{tcolorbox}[title=Example 1: Show 2. De Morgan´s Law with truth table]
\textbf{Exercise:}  
\begin{center}
Show that $\neg (p\lor q) $ and $\neg p \land \neg q$  are logically equivalent (using truth table) 
\end{center}
\textbf{Truth table:}
\begin{center}
\rowcolors{2}{gray!10}{white}
\begin{tabular}{|c|c!{\vrule width 1.5pt}c|c!{\vrule width 1.5pt}c|c|c|}
\hline
\rowcolor{gray!20}
$p$ & $q$ & $p \lor q$ & $\neg (p \lor q)$ & $\neg p$ & $\neg q$ & $\neg p \land \neg q$ \\
\hline
T & T & T & F & F & F & F \\
T & F & T & F & F & T & F \\
F & T & T & F & T & F & F \\
F & F & F & T & T & T & T \\
\hline
\end{tabular}    
\end{center}

\textbf{Conclusion:}  
Because the truth values of the compound propositions $\neg (p\lor q) $ and $\neg p \land \neg q$ agree for all possible combinations
of the truth values of $p$ and $q$, it follows that $\neg (p\lor q) \leftrightarrow \neg p \land \neg q$  is a tautology and
that these compound propositions are logically equivalent.
\end{tcolorbox}

\begin{tcolorbox}[title=Example 2: Show conditional-disjunction equivalence]
\textbf{Exercise:}  
\begin{center}
Show that $p\rightarrow q $ and $\neg p \lor q$  are logically equivalent (using truth table) 
\end{center}
\textbf{Truth table:}
\begin{center}
\rowcolors{2}{gray!10}{white}
\begin{tabular}{|c|c!{\vrule width 1.5pt}c|c|c|}
\hline
\rowcolor{gray!20}
$p$ & $q$ & $\neg p$ & $\neg p \lor q$ & $p \rightarrow q$ \\
\hline
T & T & T & F & F   \\
T & F & T & F & F   \\
F & T & T & F & T   \\
F & F & F & T & T   \\
\hline
\end{tabular}    
\end{center}

\textbf{Conclusion:}  
Because the truth values of the compound propositions $\neg (p\lor q) $ and $\neg p \land \neg q$ agree for all possible combinations
of the truth values of $p$ and $q$, it follows that $\neg (p\lor q) \leftrightarrow \neg p \land \neg q$  is a tautology and
that these compound propositions are logically equivalent.
\end{tcolorbox}
\begin{tcolorbox}[colback=white, colframe=gray!60, title=Remark 2]
If we establish a logical equivalence of two compound propositions involving three
different propositional variables $p$, $q$, and $r$. To use a truth table to establish such a logical
equivalence, we need eight rows, one for each possible combination of truth values of these
three variables. We symbolically represent these combinations by listing the truth values of $p$, $q$, and $r$, respectively.
\end{tcolorbox}
\begin{tcolorbox}[colback=white, colframe=gray!60, title=Remark 3]
These eight combinations of truth values are \textbf{TTT}, \textbf{TTF}, \textbf{TFT}, \textbf{TFF}, \textbf{FTT},
\textbf{FTF}, \textbf{FFT}, and \textbf{FFF}; we use this order when we display the rows of the truth table. Note that we
need to double the number of rows in the truth tableswe use to show that compound propositions
are equivalent for each additional propositional variable, so that 16 rows are needed to establish
the logical equivalence of two compound propositions involving four propositional variables,
and so on.
\end{tcolorbox}
\begin{tcolorbox}[colback=white, colframe=gray!60, title=Remark 4]
In general, $2^n$ rows are required if a compound proposition involves n propositional
variables. Because of the rapid growth of $2^n$, more efficient ways are needed to establish logical
equivalences, such as by using ones we already know
\end{tcolorbox}

\begin{tcolorbox}[title=Example 3: Distributive law
of disjunction over conjunction]
\textbf{Exercise:}  
\begin{center}
Show that $p\lor (q\land r) $ and $(p\lor q) \land (p \lor r)$  are logically equivalent (using truth table) 
\end{center}
\textbf{Truth table:}
\begin{center}
\rowcolors{2}{gray!10}{white}
\begin{tabular}{|c|c|c!{\vrule width 1.5pt}c|c!{\vrule width 1.5pt}c|c|c|}
\hline
\rowcolor{gray!20}
$p$ & $q$ & $r$ & $q \land r$ & $p \lor (q \land r)$ & $p \lor q$ & $p \lor r$ & $(p\lor q) \land (p\lor r)$ \\
\hline
T & T & T & T & T & T & T & T \\
\hline
T & T & F & F & T & T & T & T \\
\hline
T & F & T & F & T & T & T & T \\
\hline
T & F & F & F & T & T & T & T \\
\hline
F & T & T & T & T & T & T & T \\
\hline
F & T & F & F & F & T & F & F \\
\hline
F & F & T & F & F & T & F & F
\\
\hline
F & F & F & F & F & F & F & F \\ \hline
\end{tabular}    
\end{center}
\end{tcolorbox}

\begin{table}[h!]
\centering
\renewcommand{\arraystretch}{1.3} % Adjust vertical spacing
\rowcolors{2}{white}{gray!10} % Alternate: white, gray
\caption*{\textbf{Logical Equivalences: Logic, Programming, and Law Names}}
\begin{tabular}{|
    >{\centering\arraybackslash}m{5cm}
    |>{\centering\arraybackslash}m{7cm}
    |>{\centering\arraybackslash}m{4cm}|}
\hline
\rowcolor{gray!20}
\textbf{Equivalence (Logic)} & \textbf{Equivalence (Programming)} &\textbf{Name} \\
\hline
$p \lor \mathbf{F} \equiv p$ & p || false == p & Identity law (OR) \\
$p \land \mathbf{T} \equiv p$ & p \&\& true == p &Identity law (AND) \\
\hline
$p \lor \mathbf{T} \equiv \mathbf{T}$ & p || true == true & Domination law (OR) \\
$p \land \mathbf{F} \equiv \mathbf{F}$ & p \&\& false == false &  Domination law (AND) \\
\hline
$p \lor p \equiv p$ & p || p == p &  Idempotent law (OR) \\
$p \land p \equiv p$ & p \&\& p == p & Idempotent law (AND) \\
\hline
$\neg(\neg p) \equiv p$ & !!p == p & Double negation law \\
\hline
$p \lor q \equiv q \lor p$ & p || q == q || p & Commutative law (OR) \\
$p \land q \equiv q \land p$ & p \&\& q == q \&\& p & Commutative law (AND) \\
\hline
$(p \lor q) \lor r \equiv p \lor (q \lor r)$ & (p || q) || r == p || (q || r) & Associative law (OR) \\
$(p \land q) \land r \equiv p \land (q \land r)$ & (p \&\& q) \&\& r == p \&\& (q \&\& r) & Associative law (AND)  \\
\hline
$p \lor (q \land r) \equiv (p \lor q) \land (p \lor r)$ & p || (q \&\& r) == (p || q) \&\& (p || r)  & Distributive law (OR over AND) \\
$p \land (q \lor r) \equiv (p \land q) \lor (p \land r)$ & p \&\& (q || r) == (p \&\& q) || (p \&\& r) & Distributive law (AND over OR) \\
\hline
$\neg(p \land q) \equiv \neg p \lor \neg q$ & !(p \&\& q) == !p || !q &  De Morgan’s law (NOT AND $\rightarrow $ OR) \\
$\neg(p \lor q) \equiv \neg p \land \neg q$ & !(p || q) == !p \&\& !q & De Morgan’s law (NOT OR $\rightarrow$ AND) \\
\hline
$p \lor (p \land q) \equiv p$ & p || (p \&\& q) == p & Absorption law (OR) \\
$p \land (p \lor q) \equiv p$ & p \&\& (p || q) == p & Absorption law (AND) \\
\hline
$p \lor \neg p \equiv \mathbf{T}$ & p || !p == true   & Negation law (OR) \\
$p \land \neg p \equiv \mathbf{F}$ & p \&\& !p == false & Negation law (AND) \\
\hline
\end{tabular}
\end{table}

\newpage
\begin{table}[h!]
\centering
\renewcommand{\arraystretch}{1.3} % Adjust vertical spacing
\rowcolors{2}{white}{gray!10} % Alternate: white, gray
\caption*{\textbf{Logical Equivalences Involving Conditional Statements}}
\begin{tabular}{|
    >{\centering\arraybackslash}m{5cm}
    |>{\centering\arraybackslash}m{7cm}
    |>{\centering\arraybackslash}m{4cm}|}
\hline
\rowcolor{gray!20}
\textbf{Equivalence (Logic)} & \textbf{Equivalence (Programming)} &\textbf{Name} \\
\hline
$p \rightarrow q \equiv \neg p \lor q$ & !p || q & Conditional equivalence \\
\hline
$p \rightarrow q \equiv \neg q \rightarrow \neg p$ & !p || q == !q || !p & Contrapositive equivalence \\
\hline
$p \lor q \equiv \neg p \rightarrow q$ & !p || q & Implication from disjunction \\
\hline
$p \land q \equiv \neg(p \rightarrow \neg q)$ & !(p => !q) or !( !p || !q ) & Conjunction via implication \\
\hline
$\neg (p \rightarrow q) \equiv p \land \neg q$ & !( !p || q ) == p \&\& !q & Negation of implication \\
\hline
$(p\rightarrow q) \land (p \rightarrow r) \equiv p \rightarrow (q \land r)$ & (!p || q) \&\& (!p || r) == !p || (q \&\& r) & Distributive implication (AND) \\
\hline
$(p\rightarrow r) \land (q \rightarrow r) \equiv (p \lor q) \rightarrow r$ & (!p || r) \&\& (!q || r) == !(p || q) || r & Factoring implication (common result) \\
\hline
$(p\rightarrow q) \lor (p \rightarrow r) \equiv p \rightarrow (q \lor r)$ & (!p || q) || (!p || r) == !p || (q || r) & Distributive implication (OR) \\
\hline
$(p\rightarrow r) \lor (q \rightarrow r) \equiv (p \land q) \rightarrow r$ & (!p || r) || (!q || r) == !(p \&\& q) || r & Factoring implication (common antecedent) \\
\hline
\end{tabular}
\end{table}

\begin{table}[h!]
\centering
\renewcommand{\arraystretch}{1.3} % Adjust vertical spacing
\rowcolors{2}{white}{gray!10} % Alternate: white, gray
\caption*{\textbf{Logical
Equivalences Involving
Biconditional Statements}}
\begin{tabular}{|
    >{\centering\arraybackslash}m{5cm}
    |>{\centering\arraybackslash}m{5cm}
    |>{\centering\arraybackslash}m{6cm}|}
\hline
\rowcolor{gray!20}
\textbf{Equivalence (Logic)} & \textbf{Equivalence (Programming)} &\textbf{Name} \\
\hline
$p \leftrightarrow q \equiv (p \rightarrow q) \land (q \rightarrow p)$ &  (!p || q) \&\& (!q || p) & Biconditional as conjunction of implications   \\
\hline
$p \leftrightarrow q \equiv \neg q \leftrightarrow \neg p$ & !q == !p & Invariance under negation (logical symmetry)  \\
\hline
$p \leftrightarrow q \equiv (p \land q) \lor (\neg p \land \neg q)$ & (p \&\& q) || (!p \&\& !q) & Biconditional as agreement in truth value  \\
\hline
$\neg (p \leftrightarrow q) \equiv p \leftrightarrow \neg q$ & !(p == q) == (p == !q) & Negation of biconditional / XOR form \\
\hline
\end{tabular}
\end{table}


\begin{tcolorbox}[colback=white, colframe=gray!60, title=Remark 5]
The associative law for disjunction shows that the expression $p \lor q \lor r$ is well defined, in
the sense that it does not matter whether we first take the disjunction of $p$ with $q$ and then
the disjunction of $p\lor q$ with $r$, or if we first take the disjunction of $q$ and $r$ and then take the
disjunction of $p$ with $p \lor r$.
\end{tcolorbox}
\begin{tcolorbox}[colback=white, colframe=gray!60, title=Remark 6]
Similarly, the expression $p \land q \land r$ is well defined. By extending
this reasoning, it follows that $p_1 \lor p_2  \lor \ldots \lor p_n$ and $p_1 \land p_2 \land \ldots \land p_n$ are well defined whenever
$p_1, p_2,\ldots, p_n$ are propositions.
\end{tcolorbox}

\newpage
\begin{table}[h!]
\centering
\renewcommand{\arraystretch}{1.3} % Adjust vertical spacing
\rowcolors{2}{white}{gray!10} % Alternate: white, gray
\caption*{\textbf{Extended Version of De Morgan’s Laws}}
\begin{tabular}{|
    >{\centering\arraybackslash}m{7cm}
    |>{\centering\arraybackslash}m{4.5cm}
    |>{\centering\arraybackslash}m{5cm}|}
\hline
\rowcolor{gray!20}
\textbf{Explicit Form} & \textbf{Compact Form} & \textbf{Name} \\
\hline
$\neg(p_1 \lor p_2 \lor \ldots \lor p_n) \equiv \neg p_1 \land \neg p_2 \land \ldots \land \neg p_n$ 
& $ \neg \left(\bigvee_{j=1}^n p_j \right) \equiv \bigwedge_{j=1}^n \neg p_j $ 
& Generalized De Morgan’s Law (OR to AND) \\
\hline
$\neg(p_1 \land p_2 \land \ldots \land p_n) \equiv \neg p_1 \lor \neg p_2 \lor \ldots \lor \neg p_n$ 
& $ \neg \left(\bigwedge_{j=1}^n p_j \right) \equiv \bigvee_{j=1}^n \neg p_j $ 
& Generalized De Morgan’s Law (AND to OR) \\
\hline
\end{tabular}
\end{table}

\begin{tcolorbox}[colback=white, colframe=gray!60, title=Remark 7]
A truth table with $2^n$ rows is needed to prove the equivalence of two compound propositions
in $n$ variables. Note that the number of rows doubles for each additional propositional variable
added
\end{tcolorbox}
\begin{tcolorbox}[colback=white, colframe=gray!60, title=Remark 8]
Because $2^n$
grows extremely rapidly as $n$ increases , the use of truth tables to establish
equivalences becomes impractical as the number of variables grows. It is quicker to use other
methods, such as employing logical equivalences that we already know.
\end{tcolorbox}
