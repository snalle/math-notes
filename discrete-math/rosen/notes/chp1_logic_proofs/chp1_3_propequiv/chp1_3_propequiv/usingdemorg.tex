\begin{tcolorbox}[colback=white, colframe=gray!60, title=Remark 1]
When using De
Morgan’s laws,
remember to change
the logical connective
after you negate.
\end{tcolorbox}
\begin{tcolorbox}[colback=white, colframe=gray!60, title=Remark 2]
The two logical equivalences known as De Morgan’s laws are particularly important. They tell
us how to negate conjunctions and how to negate disjunctions.
\end{tcolorbox}
\begin{tcolorbox}[colback=white, colframe=gray!60, title=Remark 3]
In particular, the equivalence $\neg (p \lor q) \equiv \neg \land \neg q$ tells us that the negation of a disjunction is formed by taking the conjunction
of the negations of the component propositions.
\end{tcolorbox}
\begin{tcolorbox}[colback=white, colframe=gray!60, title=Remark 4]
Similarly, the equivalence $\neg (p \land q) \equiv \neg p \lor \neg q $
tells us that the negation of a conjunction is formed by taking the disjunction of the negations
of the component propositions.
\end{tcolorbox}

\begin{tcolorbox}[title=Example 1: Using De Morgan's Laws]
\textbf{Exercise:}  
\\ Use De Morgan’s laws to express the negations of:  
\begin{center}
\textit{“Miguel has a cellphone and he has a laptop computer”}  \\
\textit{“Heather will go to the concert or Steve will go to the concert”}    
\end{center}

\textbf{Let:}  
\begin{itemize}
    \item \textbf{p:} Miguel has a cellphone
    \item \textbf{q:} Miguel has a laptop computer
    \item \textbf{r:} Heather will go to the concert
    \item \textbf{s:} Steve will go to the concert
\end{itemize}

\textbf{Negation 1:}   
\begin{align*}
&\text{\textbf{Original}: } p \land q \quad \text{(Miguel has a cellphone and a laptop)} \\
&\text{\textbf{Negation}: } \neg(p \land q) \\
&\text{\textbf{Apply De Morgan's Law}: } \neg(p \land q) \equiv \neg p \lor \neg q \\
&\text{\textbf{English}: } \textit{Miguel does not have a cellphone or he does not have a laptop.}
\end{align*}
\textbf{Negation 2:}   
\begin{align*}
&\text{\textbf{Original}: } r \lor s \quad \text{(Heather or Steve will go to the concert)} \\
&\text{\textbf{Negation}: } \neg(r \lor s) \\
&\text{\textbf{Apply De Morgan's Law:} } \neg(r \lor s) \equiv \neg r \land \neg s \\
&\text{\textbf{English:} } \textit{Heather will not go to the concert and Steve will not go to the concert.}
\end{align*}
\end{tcolorbox}



