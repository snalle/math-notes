\begin{center}
\noindent\fbox{
  \parbox{0.95\linewidth}{
    \textbf{Definition:} \textbf{Tautology}. A compound proposition that is always true, no matter what the truth values of the propositional
variables that occur in it, is called a tautology.
  }
}
\end{center}
\begin{center}
\noindent\fbox{
  \parbox{0.95\linewidth}{
    \textbf{Definition:} \textbf{Contradiction.} A compound proposition that is always
false is called a contradiction
  }
}
\end{center}
\begin{center}
\noindent\fbox{
  \parbox{0.95\linewidth}{
    \textbf{Definition:} \textbf{Contingency.} A compound proposition that is neither a tautology nor a contradiction
is called a contingency
  }
}
\end{center}

\begin{tcolorbox}[title=Example 1: Examples of tautologies and contradictions using just one propositional variable]
\textbf{Exercise:}  
\begin{center}
Consider the truth tables of $p\lor \neg p$ and $p \land \neg p$ 
\end{center}

\textbf{Strategy:}  
Remember the conjunction ($\land$) and disjunction ($\lor$) operators and write first truth and false for the propositional variable $p$ and then for its negation $\neg p$.

\vspace{5pt}
\textbf{Truth Table:}
\begin{center}
\rowcolors{2}{gray!10}{white}
\begin{tabular}{|c!{\vrule width 1pt}c|c!{\vrule width 1.5pt}c|c|}
\hline
\rowcolor{gray!20}
\textbf{$p$} & \textbf{$\neg p$} & \textbf{$p \lor \neg p$} & \textbf{$p \land \neg p$} \\
\hline
T & F & T & F \\
F & T & T & F \\
\hline
\end{tabular}
\end{center}

\vspace{5pt}
\textbf{Conclusion:}   \\
The compound proposition $p \lor \neg p$ is a \textbf{tautology} — always true. \\
The compound proposition $p \land \neg p$ is a \textbf{contradiction} — always false.
\end{tcolorbox}




