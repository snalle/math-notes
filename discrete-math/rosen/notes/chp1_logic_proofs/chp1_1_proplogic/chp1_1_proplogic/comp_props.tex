\begin{itemize}
    \item \textbf{Compound propositions} We can use these connectives to build up complicated compound propositions involving any number of propositional
variables
    \item We can use truth tables to determine the truth values of these compound propositions
\end{itemize}
\begin{tcolorbox}[title=Example of compound propositions]
\textbf{Compound proposition:}  
\begin{center}
$(p \lor \neg q ) \rightarrow (p \land q)$
\end{center}
\textbf{Translated into and read as}
\begin{center}
\textit{If either $p$ is true or $q$ is not true, then both $p$ and $q$ are true.}
\end{center}
\textbf{How to solve (through truth table)?} \\
Because this truth table involves two propositional variables $p$ and $q$, there are four
rows in this truth table, one for each of the pairs of truth values TT, TF, FT, and FF \\
\textbf{Coloumn 1 and 2:} Used for the truth values of $p$ and $q$, respectively. \\
\textbf{Coloumn 3:} We find the truth value of $\neg q$ \\
\textbf{Coloum 4:} We find the truth value of $p \lor \neg q$ \\
\textbf{Coloumn 5:} We find the truth value of $p \land q$ . \\
\textbf{Coloumn 6:} We find the truth value of the original compound proposition $(p \lor \neg q ) \rightarrow (p \land q)$
\end{tcolorbox}
\begin{table}[h!]
\centering
\caption*{\textbf{Truth Table for $(p \lor \neg q) \rightarrow (p \land q)$}}
\rowcolors{2}{gray!10}{white}
\begin{tabular}{|c|c!{\vrule width 1pt}c|c|c|c!{\vrule width 1.5pt}c|}
\hline
\rowcolor{gray!20}
$p$ & $q$ & $\neg q$ & $p \lor \neg q$ & $p \land q$ & $(p \lor \neg q) \rightarrow (p \land q)$ \\
\hline
T & T & F & T & T & T \\
T & F & T & T & F & F \\
F & T & F & F & F & T \\
F & F & T & T & F & F \\
\hline
\end{tabular}
\end{table}
\subsection*{Precedence of Logical Operators (pg. 11)}
\begin{tcolorbox}[colback=white, colframe=gray!60, title=Remark 1]
We will generally use parentheses to specify the order in which logical operators
in a compound proposition are to be applied. 
\\ \\ For instance:
\begin{center}
$(p \lor q) \land (\neg r)$    
\end{center}
is the conjunction of $p \lor q$ and $\neg r$.
\\ \\
\textbf{Translation:}
\begin{center}
\textit{$p$ or $q$ and not $r$} (logic) \\
\textit{Either $p$ is true or $q$ is true, and $r$ is not true} (plain English)
\end{center}
\end{tcolorbox}
\begin{tcolorbox}[colback=white, colframe=gray!60, title=Remark 2]
However, to reduce the number of parentheses, we specify that the negation
operator is applied before all other logical operators.
\\ \\
\textbf{Example:}
\begin{center}
$\neg p \land q$ can be seen as $(\neg p) \land q$    
\end{center}
\end{tcolorbox}
\begin{tcolorbox}[colback=white, colframe=gray!60, title=Remark 3]
Another general rule of precedence is that the conjunction operator takes precedence over
the disjunction operator
\\ \\
\textbf{Example:}
\begin{center}
$p \lor q \land r$ means $p \lor (q \land r)$ \\ 
$p \land q \lor r$ means $(p \land q) \lor r$
\end{center}
\end{tcolorbox}
\begin{tcolorbox}[colback=white, colframe=gray!60, title=Remark 4]
Finally, it is an accepted rule that the conditional and biconditional operators, $\rightarrow$ and $\leftrightarrow$,
have lower precedence than the conjunction and disjunction operators
\\ \\
\textbf{Example:}
\begin{center}
$p\rightarrow q \lor r $ means $p\rightarrow (q \lor r) $  \\ 
$p \lor q \rightarrow r$ means $(p \lor q)\rightarrow r$
\end{center}
\end{tcolorbox}
\begin{tcolorbox}[colback=white, colframe=gray!60, title=Remark 5]
We will use parentheses when the order of the conditional operator and biconditional
operator is at issue, although the conditional operator has precedence over the
biconditional operator
\end{tcolorbox}
\begin{table}[h!]
\centering
\caption*{\textbf{Precedence of Logical Operators (from highest to lowest)}}
\rowcolors{2}{gray!10}{white}
\begin{tabular}{|c|c|l|}
\hline
\rowcolor{gray!20}
\textbf{Precedence} & \textbf{Symbol} & \textbf{Operator Name} \\
\hline
1 (highest) & $\neg$ & Negation (Not) \\
2 & $\land$ & Conjunction (And) \\
3 & $\lor$ & Disjunction (Or) \\
4 & $\oplus$ & Exclusive Or (XOR) \\
5 & $\rightarrow$ & Conditional (Implication) \\
6 & $\leftrightarrow$ & Biconditional (If and only if) \\
\hline
\end{tabular}
\end{table}













