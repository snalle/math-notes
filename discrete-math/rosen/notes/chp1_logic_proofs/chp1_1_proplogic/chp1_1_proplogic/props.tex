\vspace{5pt}
\begin{center}
\noindent\fbox{
  \parbox{0.95\linewidth}{
    \textbf{Definition:} A proposition is a declarative sentence that is either true or false, but not both.
  }
}
\end{center}
\vspace{5pt}
\begin{itemize}
    \item \textbf{Examples of propositions:}
    \begin{itemize}
        \item \textit{"Washington, D.C., is the capital of the United States of America."}, which is true.
        \item \textit{"Toronto is the capital of Canada."}, which is false.
        \item $1+1=2$, which is true.
        \item $2+2=3$, which is false.
    \end{itemize}
    \item \textbf{Examples of sentence that are NOT proposition:}
    \begin{itemize}
        \item \textit{"What time is it?"}, which is a question.
        \item \textit{"Read this carefully."}, which is a command.
        \item $x+1=2$, which is neither true or false because we have not assigned a value or values to $x$
        \item $x+y=z$, which is neither true or false because we have not assigned a value or values to $x$, $y$ or $z$.
    \end{itemize}
    \item Use letters to denote propositional variables(or sentential variables), that is, variables
that represent propositions, just as letters are used to denote numerical variables.
    \item The conventional letters used for propositional variables are p, q, r, s, ...
    \item \textbf{Truth values:} The truth value of a proposition is true, denoted by T, if it is a true proposition, and the truth value of a proposition is false, denoted by F, if it is a false proposition. 
    \item Propositions that cannot be expressed in terms of simpler propositions are called atomic propositions.
    \item New propositions, called compound propositions, are formed from existing propositions using logical operators.
\end{itemize}