\begin{center}
\noindent\fbox{
  \parbox{0.95\linewidth}{
    \textbf{Definition:} A \textbf{bit} is a symbol with two possible values, namely,
0 (zero) and 1 (one).
  }
}
\end{center}
\begin{tcolorbox}[colback=white, colframe=gray!60, title=Remark 1]
Computers represent information using bits. \\ \\
This meaning of the word bit comes from binary digit, because zeros and
ones are the digits used in binary representations of numbers
\end{tcolorbox}
\begin{tcolorbox}[colback=white, colframe=gray!60, title=Remark 2]
A bit can be used to represent a truth value, because
there are two truth values, namely, true and false. \\ As is customarily done, we will use a 1 bit
to represent true and a 0 bit to represent false. That is, 1 represents T (true), 0 represents F
(false).
\end{tcolorbox}
\begin{center}
\noindent\fbox{
  \parbox{0.95\linewidth}{
    \textbf{Definition:} A variable is called a \textbf{Boolean variabl}e if its value is either true or false.
  }
}
\end{center}
\begin{tcolorbox}[colback=white, colframe=gray!60, title=Remark 3]
Consequently, a Boolean variable can be represented using a bit.
\end{tcolorbox}
\begin{tcolorbox}[colback=white, colframe=gray!60, title=Remark 4]
Computer bit operations correspond to the logical connectives. By replacing true by a one
and false by a zero in the truth tables for the operators: $\land , \lor$ and $\oplus$.
\\ \\ 
We will also use the notation \textbf{OR}, \textbf{AND}, and \textbf{XOR} for the operators as is done in various programming languages
\end{tcolorbox}
\begin{table}[h!]
\centering
\begin{minipage}{0.48\linewidth}
\centering
\caption*{\textbf{Bitwise Table (0/1)}}
\rowcolors{2}{gray!10}{white}
\begin{tabular}{|c|c!{\vrule width 1pt}c|c|c|}
\hline
\rowcolor{gray!20}
$p$ & $q$ & $p \lor q$ & $p \land q$ & $p \oplus q$ \\
\hline
0 & 0 & 0 & 0 & 0 \\
0 & 1 & 1 & 0 & 1 \\
1 & 0 & 1 & 0 & 1 \\
1 & 1 & 1 & 1 & 0 \\
\hline
\end{tabular}
\end{minipage}
\hfill
\begin{minipage}{0.48\linewidth}
\centering
\caption*{\textbf{Truth Table (T/F)}}
\rowcolors{2}{gray!10}{white}
\begin{tabular}{|c|c!{\vrule width 1pt}c|c|c|}
\hline
\rowcolor{gray!20}
$p$ & $q$ & $p \lor q$ & $p \land q$ & $p \oplus q$ \\
\hline
F & F & F & F & F \\
F & T & T & F & T \\
T & F & T & F & T \\
T & T & T & T & F \\
\hline
\end{tabular}
\end{minipage}
\end{table}
\begin{center}
\noindent\fbox{
  \parbox{0.95\linewidth}{
    \textbf{Definition:} A \textbf{bit string} is a sequence of zero or more bits. The length of this string is the number of bits
in the string.
  }
}
\end{center}
\begin{tcolorbox}[colback=white, colframe=gray!60, title=Remark 5]
Information is often represented using bit strings, which are lists of zeros and ones. \\ When
this is done, operations on the bit strings can be used to manipulate this information.
\end{tcolorbox}
\begin{tcolorbox}[colback=white, colframe=gray!60, title=Remark 6]
We define the \textbf{bitwise OR}, \textbf{bitwise AND}, and \textbf{bitwise XOR} of two strings of the same length to be the strings that have as their bits the OR, AND, and XOR of the corresponding bits in the two strings, respectively
\end{tcolorbox}
\begin{tcolorbox}[title=Example of bitwise operations]
Find the \textbf{bitwise OR}, \textbf{bitwise AND}, and b\textbf{itwise XOR} of the bit strings
\begin{center}
01 1011 0110 and 
\\ 11 0001 1101
\end{center}
(Here, and throughout this book, bit strings will be split into blocks of four bits
to make them easier to read.)
\\ \\
\textbf{\underline{Solution}:} \\
The bitwise OR, bitwise AND, and bitwise XOR of these strings are obtained by taking
the OR, AND, and XOR of the corresponding bits, respectively \\ \\
\textbf{Bitwise OR:} \\
To compute the \textbf{bitwise OR}, we compare each corresponding pair of bits from A and B.  
The OR operation results in 1 if \textit{either} of the bits is 1, and 0 only if both are 0.
\[
\begin{array}{c c c c c c c c c c c c c}
A & = & 0 & 1 & 1 & 0 & 1 & 1 & 0 & 1 & 1 & 0 \\
B & = & 1 & 1 & 0 & 0 & 0 & 1 & 1 & 1 & 0 & 1 \\
\hline
A \text{ OR } B & = & 1 & 1 & 1 & 0 & 1 & 1 & 1 & 1 & 1 & 1
\end{array}
\]
\textbf{Bitwise AND:} \\
To compute the \textbf{bitwise AND}, we compare each corresponding pair of bits from A and B.  
The AND operation results in 1 only if \textit{both} bits are 1; otherwise, it is 0.
\[
\begin{array}{c c c c c c c c c c c c c}
A & = & 0 & 1 & 1 & 0 & 1 & 1 & 0 & 1 & 1 & 0 \\
B & = & 1 & 1 & 0 & 0 & 0 & 1 & 1 & 1 & 0 & 1 \\
\hline
A \text{ AND } B & = & 0 & 1 & 0 & 0 & 0 & 1 & 0 & 1 & 0 & 0
\end{array}
\]
\textbf{Bitwise XOR:} \\
To compute the \textbf{bitwise XOR}, we compare each corresponding pair of bits from A and B.  
The XOR operation results in 1 if the bits are \textit{different}, and 0 if they are the same.
\[
\begin{array}{c c c c c c c c c c c c c}
A & = & 0 & 1 & 1 & 0 & 1 & 1 & 0 & 1 & 1 & 0 \\
B & = & 1 & 1 & 0 & 0 & 0 & 1 & 1 & 1 & 0 & 1 \\
\hline
A \text{ XOR } B & = & 1 & 0 & 1 & 0 & 1 & 0 & 1 & 0 & 1 & 1
\end{array}
\]
\textbf{\underline{Full solution}:}
\begin{center}
\begin{tabular}{>{\bfseries}r l}
Bit String A:   & 01\ 1011\ 0110 \\
Bit String B:   & 11\ 0001\ 1101 \\
\hline
Bitwise OR:     & 11\ 1011\ 1111 \\
Bitwise AND:    & 01\ 0001\ 0100 \\
Bitwise XOR:    & 10\ 1010\ 1011 \\
\end{tabular}
\end{center}





\end{tcolorbox}




