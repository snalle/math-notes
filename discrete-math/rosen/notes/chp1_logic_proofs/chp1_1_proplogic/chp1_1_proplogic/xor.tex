\vspace{5pt}
\tcbset{colback=gray!5, colframe=black!70, boxrule=0.5pt, arc=2pt, left=6pt, right=6pt, top=4pt, bottom=4pt}
\begin{tcolorbox}[title=Definition: Exclusive Or (XOR)]
Let $p$ and $q$ be propositions. The exclusive or of $p$ and $q$, denoted by $p \oplus q$, is the proposition:  
\begin{center}
\textit{"$p$ or $q$, but not both."}
\end{center}
The exclusive or $p \oplus q$ is true when exactly one of $p$ or $q$ is true, and false otherwise.
\end{tcolorbox}
\begin{table}[h!]
\centering
\caption*{\textbf{Common Notations for Exclusive Or (XOR) and Their Contexts}}
\rowcolors{2}{gray!10}{white}
\begin{tabular}{|c|l|l|}
\hline
\rowcolor{gray!20}
\textbf{Notation} & \textbf{Read as} & \textbf{Commonly Used In} \\
\hline
$p \oplus q$ & $p$ XOR $q$ & Mathematical logic, discrete math, CS theory \\
$p \veebar q$ & $p$ exclusive or $q$ & Set theory, logic (less common) \\
$p \neq q$ & $p$ not equal to $q$ & Logic circuits (when $p$, $q$ are Boolean) \\
$p \texttt{\^{} } q$ & $p$ XOR $q$ & Programming (Python, C++, JavaScript — bitwise XOR) \\
$\texttt{xor}(p, q)$ & XOR function of $p$ and $q$ & Pseudocode, some programming languages and logic textbooks \\
\hline
\end{tabular}
\end{table}
\begin{table}[h!]
\centering
\caption*{\textbf{Examples of Propositions and Their Exclusive Or (XOR)}}
\rowcolors{2}{gray!10}{white}
\begin{tabular}{|p{4.5cm}|p{4.5cm}|p{6cm}|}
\hline
\rowcolor{gray!20}
\textbf{Proposition $p$} & \textbf{Proposition $q$} & \textbf{Exclusive Or $p \oplus q$ (Plain English)} \\
\hline
A student can have a salad with dinner. & A student can have soup with dinner. & A student can have soup or salad, but not both, with dinner. \\
I will use all my savings to travel to Europe. & I will use all my savings to buy an electric car. & I will use all my savings to travel to Europe or to buy an electric car, but not both. \\
\hline
\end{tabular}
\end{table}
\begin{table}[h!]
\centering
\caption*{\textbf{Truth Table for the Exclusive Or $p \oplus q$}}
\rowcolors{2}{gray!10}{white}
\begin{tabular}{|c|c!{\vrule width 1.5pt}c|}
\hline
\rowcolor{gray!20}
\textbf{$p$} & \textbf{$q$} & \textbf{$p \oplus q$} \\
\hline
T & T & F \\
T & F & T \\
F & T & T \\
F & F & F \\
\hline
\end{tabular}
\end{table}
\begin{tcolorbox}[colback=white, colframe=gray!60, title=Remark 1]
The exclusive or $p \oplus q$ is true only when exactly one of $p$ or $q$ is true. It is false when both are true or both are false.
\end{tcolorbox}
\begin{tcolorbox}[colback=white, colframe=gray!60, title=Remark 2]
In Boolean logic and programming, XOR is often used to represent binary decisions where two conditions are mutually exclusive.
\end{tcolorbox}


