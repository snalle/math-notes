\begin{tcolorbox}[title=Example 1: King's Daughter Treasure Logic Puzzle (1/2)]
\textbf{Puzzle:}  
\begin{itemize}
    \item As a reward for saving his daughter from pirates, the King has given you the opportunity to win
a treasure hidden inside one of three trunks. 
    \item The two trunks that do not hold the treasure are
empty. 
    \item To win, you must select the correct trunk. \item Trunks 1 and 2 are each inscribed with the
message “\textit{This trunk is empty,}” and Trunk 3 is inscribed with the message “\textit{The treasure is in
Trunk 2.}” 
    \item The Queen, who never lies, tells you that only one of these inscriptions is true, while
the other two are wrong. 
    \item Which trunk should you select to win?
\end{itemize}{}

\textbf{Strategy:}  
Let $p_i$ be the proposition that the treasure is in Trunk $i$, for $i = 1, 2, 3$. To translate into
propositional logic the Queen’s statement that exactly one of the inscriptions is true, we observe
that the inscriptions on Trunk 1, Trunk 2, and Trunk 3, are:
\begin{center}
$\neg p_1 \ \ \  \text{(Not in Trunk 1)} $ \\
$\neg p_2 \ \ \  \text{(Not in Trunk 2)} $ \\
$p_2 \ \ \  \text{(In Trunk 2)} $ \\
\end{center}
Then we have to construct the 3 different scenarios:
\begin{itemize}
    \item Inscription in Trunk 1 is correct:
    \begin{center}
$\neg p_1 \land \neg(\neg p_2) \land \neg p_2$    
\end{center}
   \item Inscription in Trunk 2 is correct:
    \begin{center}
$\neg (\neg p_1) \land \neg p_2 \land \neg p_2$    
\end{center}
    \item Inscription in Trunk 3 is correct:
    \begin{center}
$\neg (\neg p_1) \land \neg (\neg p_2) \land p_2$    
\end{center}
\end{itemize}
Only one scenario can be correct, so we have to combine them with the disjunction operator $\lor$ and we get:
\begin{center}
$\left( \neg p_1 \land \neg(\neg p_2) \land \neg p_2 \right) \lor \left(\neg (\neg p_1) \land \neg p_2 \land \neg p_2 \right) \lor \left(\neg (\neg p_1) \land \neg (\neg p_2) \land p_2 \right) $    
\end{center}
\textbf{Reducing and simplifying (1/2):}
\begin{center}
$\neg (\neg p_i) = p_i $    
\end{center}
\begin{center}
$\left( \neg p_1 \land p_2 \land \neg p_2 \right) \lor \left(p_1 \land \neg p_2 \land \neg p_2 \right) \lor \left(p_1 \land  p_2 \land p_2 \right) $    
\end{center}
\begin{center}
$\left(p_1 \land \neg p_2  \right) \lor \left(p_1 \land  p_2\right) $    
\end{center}
Now using distributive law: $p_1 \land (\neg p_2 \lor p_2) $ 
\end{tcolorbox}
\begin{tcolorbox}[title=Example 1: King's Daughter Treasure Logic Puzzle (2/2)]
\textbf{Reducing and simplying (2/2):}  
\begin{center}
$\left(p_1 \land \neg p_2  \right) \lor \left(p_1 \land  p_2\right) = p_1 \land (\neg p_2 \lor p_2) $    
\end{center}
But we know that it has to be true that either $\neg p_2$ or $p_2$ and we can denote this as
\begin{center}
$\neg p_2 \lor p_2 = \mathbf{T}$   
\end{center}
This gives
\begin{center}
$p_1 \land (\neg p_2 \lor p_2) = p_1 \land \mathbf{T}$ 
\end{center}
But this is equivalent to
\begin{center}
$ p_1 \land \mathbf{T} = p_1$ 
\end{center}
\textbf{Conclusion:} \\
So the treasure is in Trunk 1 (that is, $p_1$ is true), and $p_2$ and $p_3$ are false; and the inscription on Trunk 2 is the only true one.
\end{tcolorbox}

\begin{tcolorbox}[title=Example 2: Raymond Smullyan´s Knights and Knaves]
\textbf{Puzzle:}  
\begin{itemize}
    \item In [Sm78] Smullyan posed many puzzles about an island that has two kinds of inhabitants,
knights, who always tell the truth, and their opposites, knaves, who always lie.
    \item You encounter two people A and B.
    \item What are A and B if A says “\textit{B is a knight}” and B says “T\textit{he two of us are
opposite types”?} 
\end{itemize}{}

\textbf{Strategy:}  
Let $p$ and $q$ be the statements that A is a knight and B is a knight, respectively, so that
\begin{center}
$\neg p \ \ \ \text{(A is a knave)} $ \\
$\neg q \ \ \ \text{(B is a knave)}$
\end{center}
Then we have to construct the 2 different scenarios:
\begin{itemize}
    \item \textbf{A is a knight and tells the truth:}\\
    Then $p$ is True, and B is also a knight ($q$ is True) and tells the truth. \\ But that means the statement $B$ says also has to be true and this cannot be, since they are the same type. So this scenario is impossible and both A and B can´t be knights
    \item \textbf{A is a knave and lies:} \\
    Then A lies about B being a knight. This means that B also must be a knave. This also works when looking at B´s statement, because B also lies that the are different types. 
\end{itemize}
\textbf{Conclusion:} \\
We can conclude that both A and B are
knaves.
 
\end{tcolorbox}

\begin{tcolorbox}[title=Example 3: Muddy children puzzle]
\textbf{Puzzle:}  
\begin{itemize}
    \item A father tells his two children, a boy and a girl, to play in their backyard without getting dirty. However, while playing, both children get mud on their foreheads
    \item When the children stop playing,
the father says “A\textit{t least one of you has a muddy forehead,}” and then asks the children to
answer “\textit{Yes}” or “\textit{No}” to the question: \textit{“Do you know whether you have a muddy forehead?}”
    \item The father asks this question twice. 
    \item What will the children answer each time this question is asked, assuming that a child can see whether his or her sibling has a muddy forehead, but cannot see his or her own forehead?
    \item Assume that both children are honest and that the children answer each question simultaneously.
\end{itemize}

\textbf{Strategy:}  
Let $s$ be the statement that the son has a muddy forehead and let $d$ be the statement that
the daughter has a muddy forehead. 
\\ \\ 
When the father says that at least one of the two children
has a muddy forehead, he is stating that the disjunction is true
\begin{center}
$s \lor d$
\end{center}
Then we have to construct the 2 different times the father asks:
\begin{itemize}
    \item \textbf{Round 1:}\\
    We assum that each children tells the true and at the same time, so they would both answer "\textit{No}", since the son only knows $d$ is true and daughter only knows $s$ is true. 
    \item \textbf{Round 2:}\\
    Each child now knows that they themselves must have mud on their forehead, since they both answered "\textit{No}" on the 1. question and the father stated that at least one of them has a muddy forehead. \\ \\ They are both going to answer "\textit{Yes}" this time.
\end{itemize}
\textbf{Conclusion:} \\
First time the father asks, both children answer "\textit{No}". \\ Second time the father asks, both children answer "\textit{Yes}".
 
\end{tcolorbox}

