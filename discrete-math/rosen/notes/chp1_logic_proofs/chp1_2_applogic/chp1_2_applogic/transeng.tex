\begin{tcolorbox}[title=Example 1: Translating from English to a Logical Expression]
\textbf{Statement:}  
\begin{center}
\textit{"You can access the Internet from campus only if you are a computer science major or you are not a freshman."}
\end{center}

\textbf{Strategy:}  
Rather than representing the entire sentence as a single propositional variable (e.g. $p$), which would not help in understanding or reasoning, we break it into smaller propositions and connect them using logical operators.

\vspace{1em}

\textbf{Let:}
\begin{itemize}
  \item $a$: You can access the Internet from campus
  \item $c$: You are a computer science major
  \item $f$: You are a freshman
\end{itemize}

\vspace{0.5em}

\textbf{Translation:}
\[
a \rightarrow (c \lor \neg f)
\]
\end{tcolorbox}
\begin{tcolorbox}[title=Example 2: Translating from English to a Logical Expression]
\textbf{Statement:}  
\begin{center}
\textit{"You cannot ride the roller coaster if you are under 4 feet tall unless you are older than 16
years old"}
\end{center}

\textbf{Strategy:}  
Same as previous Example 1. \\ \\
\textbf{Let:}
\begin{itemize}
  \item $q$: You can ride the roller coaster
  \item $r$: You are under 4 feet tall
  \item $s$: You are older than 16 years old
\end{itemize}

\vspace{0.5em}

\textbf{Translation:}
\[
(r \land \neg s) \rightarrow \neg q
\]
\textbf{Explanation:}  \\
The sentence contains an exception with "unless", which in logic means "if not".  \\
So "\textit{unless you are older than 16}" is logically equivalent to "\textit{if you are not older than 16}", or $\neg s$. \\
Thus, the phrase becomes:  
"\textit{If you are under 4 feet tall and not older than 16, then you cannot ride the roller coaster.}"
That is:
\[
(r \land \neg s) \rightarrow \neg q
\]

\end{tcolorbox}

