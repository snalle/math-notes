\begin{center}
\noindent\fbox{
  \parbox{0.95\linewidth}{
    \textbf{Definition:} \textbf{Boolean searching} refers to a search technique that uses tools called operators and modifiers to limit, widen, and refine your search results
  }
}
\end{center}
\begin{tcolorbox}[colback=white, colframe=gray!60, title=Remark 1]
In Boolean searches, the connective AND is used to match records that contain both of
two search terms, the connective OR is used to match one or both of two search terms, and the
connective NOT (sometimes written as AND NOT ) is used to exclude a particular search term
\end{tcolorbox}
\begin{tcolorbox}[colback=white, colframe=gray!60, title=Remark 2]
\textbf{Web Page Searching}: 
 Most Web search engines support Boolean searching techniques, which
is useful for finding Web pages about particular subjects
\end{tcolorbox}
\begin{tcolorbox}[title=Example 1: Boolean Searching Using Logical Operators]
\textbf{Statement:}  
\begin{center}
\textit{You want to search for Web pages about universities in New Mexico.}
\end{center}

\textbf{Strategy:}  
We can use Boolean logic operators like AND, OR, and NOT to construct search expressions that include only the pages we want. We'll analyze what happens with different Boolean formulations.

\textbf{Let:}
\begin{itemize}
    \item $N$: The page contains the word "NEW"
    \item $M$: The page contains the word "MEXICO"
    \item $U$: The page contains the word "UNIVERSITIES"
\end{itemize}
\textbf{Boolean Search:}
\[
N \land M \land U \ \ \ (\text{Logic operators})
\]
\[
\text{NEW AND MEXICO AND UNIVERSITIES}
\]

This query returns all pages that contain the words "NEW", "MEXICO", and "UNIVERSITIES".
\\
\textbf{Observation:}  
This will include:
\begin{itemize}
    \item Relevant pages about universities in New Mexico (desired)
    \item Irrelevant pages such as ones about "new universities in Mexico" (not desired)
\end{itemize}

\vspace{0.5em}

\textbf{Improved Search Strategy:}  
To reduce irrelevant results, use quotation marks to group the phrase "NEW MEXICO" as a single unit:

\[
\texttt{"NEW MEXICO"} \land U \ \ \text{(Logic operators)}
\]
\[
\text{“NEW
MEXICO” AND UNIVERSITIES.}
\]

This ensures the search engine treats "NEW MEXICO" as a proper phrase and reduces matches like “new universities in Mexico”.

\textbf{Conclusion:}  
Boolean search logic mimics propositional logic. Using grouping (e.g., quotation marks) and appropriate connectives improves search precision.
\end{tcolorbox}

\begin{tcolorbox}[title=Example 2: Boolean Searching with OR and AND Precedence]
\textbf{Statement:}  
\begin{center}
\textit{You want to search for Web pages about universities in either New Mexico or Arizona.}
\end{center}

\textbf{Strategy:}  
We use Boolean logic to capture the intent: pages that mention "universities" and either:
- both "NEW" and "MEXICO", or
- "ARIZONA".
\\ \\
\textbf{Let:}
\begin{itemize}\setlength\itemsep{0em}
    \item $N$: The page contains the word "NEW"
    \item $M$: The page contains the word "MEXICO"
    \item $A$: The page contains the word "ARIZONA"
    \item $U$: The page contains the word "UNIVERSITIES"
\end{itemize}

\textbf{Boolean Search:}
\[
((N \land M) \lor A) \land U
\]
\[
\texttt{(NEW AND MEXICO OR ARIZONA) AND UNIVERSITIES}
\]

\textbf{Note on Precedence:}  
In Boolean logic (and in most search engines), the AND operator has higher precedence than OR.  
So this is interpreted as:
\[
((\text{NEW AND MEXICO}) \text{ OR ARIZONA}) \text{ AND UNIVERSITIES}
\]

\textbf{Observation:}  
This query matches:
\begin{itemize}\setlength\itemsep{0em}
    \item Pages about universities in New Mexico (contain NEW, MEXICO, and UNIVERSITIES)
    \item Pages about universities in Arizona (contain ARIZONA and UNIVERSITIES)
    \item But also possibly irrelevant pages such as:
    \begin{itemize}\setlength\itemsep{0em}
        \item Pages with NEW and MEXICO but not universities
        \item Pages about Arizona universities not specifically comparing them with New Mexico
    \end{itemize}
\end{itemize}

\vspace{0.5em}

\textbf{Improved Search Strategy:}  
Use grouping with parentheses or quotation marks to better control matching, especially when using search engines like Google:

\[
\texttt{"NEW MEXICO" OR ARIZONA AND UNIVERSITIES}
\]

This still depends on how the search engine parses precedence. Some engines may interpret it as:
\[
\texttt{"NEW MEXICO"} \lor (\texttt{ARIZONA} \land \texttt{UNIVERSITIES})
\]
which may not be what you want.
\\ \\
\textbf{Conclusion:}  
Operator precedence matters in Boolean search. To refine results, group expressions using parentheses or quotes, and be aware that not all engines use the same precedence rules.
\end{tcolorbox}

\begin{tcolorbox}[title=Example 3: Boolean Searching Using NOT (Exclusion)]
\textbf{Statement:}  
\begin{center}
\textit{You want to find Web pages about universities in Mexico, but exclude those about New Mexico.}
\end{center}

\textbf{Strategy:}  
A naive search for \texttt{MEXICO AND UNIVERSITIES} will match both:
\begin{itemize}\setlength\itemsep{0em}
    \item Pages about universities in Mexico (desired)
    \item Pages about universities in New Mexico (undesired)
\end{itemize}

To avoid including pages that mention “NEW”, we use the \textbf{NOT} operator to exclude them.

\vspace{0.5em}

\textbf{Let:}
\begin{itemize}\setlength\itemsep{0em}
    \item $M$: The page contains the word "MEXICO"
    \item $U$: The page contains the word "UNIVERSITIES"
    \item $N$: The page contains the word "NEW"
\end{itemize}

\textbf{Boolean Search:}
\[
(M \land U) \land \neg N
\]
\[
\texttt{(MEXICO AND UNIVERSITIES) NOT NEW}
\]

\textbf{Observation:}  
This search will include:
\begin{itemize}\setlength\itemsep{0em}
    \item Pages that mention both “MEXICO” and “UNIVERSITIES”
    \item But \textbf{exclude} any pages that also contain the word “NEW” — removing false positives related to "New Mexico"
\end{itemize}

\vspace{0.5em}

\textbf{Search Engine Note:}  
Many engines like Google do not use the word \texttt{NOT}, but instead use the minus sign (\texttt{-}) to indicate exclusion.
\\ \\
\textbf{Google-style equivalent:}
\[
\texttt{MEXICO UNIVERSITIES -NEW}
\]

\textbf{Conclusion:}  
The NOT operator (or \texttt{-} in practice) is useful in Boolean searching for filtering out results that include unwanted keywords. When searching for content that overlaps with similar terms, this exclusion strategy improves result relevance.
\end{tcolorbox}
