\begin{tcolorbox}[title=Example 1: Express a specification using logical connectives.]
\textbf{Specification:}  
\begin{center}
\textit{"The automated reply cannot be sent when the file system is full"}
\end{center}
\textbf{Let:}
\begin{itemize}
  \item $p$: The automated reply can be sent
  \item $q$: The file system is full
\end{itemize}
Then we can express the negation of $p$ as $\neg p:$
\begin{center}
\textit{"It is not the case that the automated reply can be sent"} \\ or simply: \textit{"The automated reply cannot be sent"}
\end{center}{}

\vspace{0.5em}

\textbf{Translation (as conditional statement):}
\[
q \rightarrow \neg p
\]
\end{tcolorbox}
\begin{tcolorbox}[colback=white, colframe=gray!60, title=Remark 1]
System specifications should be \textbf{consistent}, that is, they should not contain conflicting requirements
that could be used to derive a contradiction. \\ When specifications are not consistent,
there would be no way to develop a system that satisfies all specifications.
\end{tcolorbox}

\begin{tcolorbox}[title=Example 2: Determine Consistency of System Specifications]
\textbf{Consider the following specifications:}
\begin{enumerate}
    \item \textit{The diagnostic message is stored in the buffer or it is retransmitted.}
    \item \textit{The diagnostic message is not stored in the buffer.}
    \item \textit{If the diagnostic message is stored in the buffer, then it is retransmitted.}
\end{enumerate}

\textbf{Let:}
\begin{itemize}
  \item $p$: The diagnostic message is stored in the buffer
  \item $q$: The diagnostic message is retransmitted
\end{itemize}

\vspace{0.5em}

\textbf{Translations:}
\begin{enumerate}
    \item $p \lor q$
    \item $\neg p$
    \item $p \rightarrow q$
\end{enumerate}

\textbf{Truth Assignment and Evaluation:}

Assume: \( p = \text{F},\ q = \text{T} \)

\begin{itemize}
    \item (1) \( p \lor q = \text{F} \lor \text{T} = \text{T} \)
    \item (2) \( \neg p = \neg \text{F} = \text{T} \)
    \item (3) \( p \rightarrow q = \text{F} \rightarrow \text{T} = \text{T} \) (a conditional with false hypothesis is always true)
\end{itemize}

\textbf{Conclusion:}

Under the assignment \( p = \text{F},\ q = \text{T} \), all three specifications evaluate to true.  \\
\textbf{Therefore, the system specifications are consistent.}

\end{tcolorbox}
\begin{tcolorbox}[title=Example 3: Determine Consistency of System Specifications (2)]
\textbf{Consider the following specifications:}
\begin{enumerate}
    \item \textit{The diagnostic message is stored in the buffer or it is retransmitted.}
    \item \textit{The diagnostic message is not stored in the buffer.}
    \item \textit{If the diagnostic message is stored in the buffer, then it is retransmitted.}
    \item \textit{The diagnostic message is not retransmitted.}
\end{enumerate}

\textbf{Let:}
\begin{itemize}
  \item $p$: The diagnostic message is stored in the buffer
  \item $q$: The diagnostic message is retransmitted
\end{itemize}

\vspace{0.5em}

\textbf{Translations:}
\begin{enumerate}
    \item $p \lor q$
    \item $\neg p$
    \item $p \rightarrow q$
    \item $\neg q$
\end{enumerate}

\textbf{Truth Assignment and Evaluation:}

Assume: \( p = \text{F},\ q = \text{T} \)

\begin{itemize}
    \item (1) \( p \lor q = \text{F} \lor \text{T} = \text{T} \)
    \item (2) \( \neg p = \neg \text{F} = \text{T} \)
    \item (3) \( p \rightarrow q = \text{F} \rightarrow \text{T} = \text{T} \) (a conditional with a false hypothesis is always true)
    \item (4) \( \neg q = \neg \text{T} = \text{F} \) \textbf{← Contradiction!}
\end{itemize}

\textbf{Conclusion:}

Under the truth assignment \( p = \text{F},\ q = \text{T} \), the first three specifications are satisfied.  
However, the fourth specification (\( \neg q \)) is false.  

\textbf{Therefore, the system specifications are inconsistent.}
\end{tcolorbox}

